%%%%%%%%%%%%%%%%%%%%%%%%%%%%%%%%%_TEMPLATE_PACKAGES_%%%%%%%%%%%%%%%%%%%%%%%%%%%%%%%%%
\documentclass[
	a4paper,
	pagesize,
	pdftex,
	12pt,
	% twoside,    % + BCOR darunter: für doppelseitigen Druck aktivieren, sonst beide deaktivieren
	% BCOR=5mm,   % Dicke der Bindung berücksichtigen (Copyshop fragen, wie viel das ist)
	ngerman,
	fleqn,
	final,
	]{scrartcl}
    
% PACKAGES FOR THE INSTITUTSVORLAGE
\usepackage[utf8]{inputenc}
\usepackage[ngerman]{babel}
\usepackage[unicode=true]{hyperref} % label, references, websites and crosslinks in PDF
\usepackage{setspace} % für Elemente der Titelseite
\usepackage{tikz} % für Elemente der Titelseite
\usepackage{tabularx} % für Elemente der Titelseite
\usepackage[draft=false,babel,tracking=true,kerning=true,spacing=true]{microtype} % optischer Randausgleich
\microtypesetup{nopatch={footnote}} % disable footnote microtype warning
% PACKAGES FOR USING THIS PREBUILD STUCTURE
\usepackage{csquotes}   % better quote style for biblatex
\usepackage{graphicx}
%%%%%%%%%%%%%%%%%%%%%%%%%%%%%%%%%_YOUR_PACKAGES_%%%%%%%%%%%%%%%%%%%%%%%%%%%%%%%%%
% UTILITY PACKAGES
\usepackage{cite}
\usepackage{comment} % enables block comments via \begin{comment} ... \end{comment} environment
\usepackage{amsthm} % for definitions, lemmas, etc. - also for defining your own stuff, eg below:
    %\theoremstyle{definition}  % defines a new theorem called definition
    %\newtheorem{definition}{Definition}[section]   % definition setup and call
% IMAGE PACKAGES
\usepackage{wrapfig}    % create figures with wrapped text around it
\usepackage{caption}    % better captions for figures
\usepackage{subcaption} % captions for subfigures
% PRESENTATION PACKAGES
\usepackage{booktabs} % for professional tables
\usepackage{longtable} % for tables over multiple pages
\usepackage{pdflscape} % enables landscape mode for multiple pages in PDF (with longtable)
\usepackage{afterpage}
\usepackage[T1]{fontenc} % clears current page by flushing all floats
%%%%%%%%%%%%%%%%%%%%%%%%%%%%%%%%%%%_TITLE_PAGE_%%%%%%%%%%%%%%%%%%%%%%%%%%%%%%%%%%%
\begin{document}

% Beispielhafte Nutzung der Vorlage für die Titelseite (bitte anpassen):
\input{Institutsvorlage}

\titel{This is a Long Title which Spans Over Several Lines and Shall Always Look the Same} % Titel der Arbeit
\typ{Masterarbeit} % Typ der Arbeit:  Diplomarbeit, Masterarbeit, Bachelorarbeit
\grad{Master of Science (M. Sc.)} % erreichter Akademischer Grad
    % z.B.: Master of Science (M. Sc.), Master of Education (M. Ed.), Bachelor of Science (B. Sc.), Bachelor of Arts (B. A.)
\autor{Maxime Mustermann} % Autor der Arbeit, mit Vor- und Nachname
\gebdatum{1.1.1970} % Geburtsdatum des Autors
\gebort{Bielefeld} % Geburtsort des Autors
\gutachter{Prof. Dr. Dr. hc. mult. Kerstin von Kienfeld}{Prof. Dr. Bernd Blume} % Erst- und Zweitgutachter der Arbeit
\mitverteidigung % entfernen, falls keine Verteidigung erfolgt
\makeTitel

\begin{abstract}
	\textbf{Abstract.} Write your abstract here.
\end{abstract}
\newpage

%%%%%%%%%%%%%%%%%%%%%%%%%%%%%%%%_TABLE_OF_CONTENTS_%%%%%%%%%%%%%%%%%%%%%%%%%%%%%%%%
\pagenumbering{gobble}  % page numbers invisible for TOC and filler pages
\tableofcontents
\cleardoublepage    % deactivate for one-sided printing
%\newpage           % activate for one-sided printing
%%%%%%%%%%%%%%%%%%%%%%%%%%%%%%%%%%%%%_CHAPTERS_%%%%%%%%%%%%%%%%%%%%%%%%%%%%%%%%%%%%%
\pagenumbering{arabic}  % start regular page numbers from here
% insert and call your designated chapters here from chapters/... folder

\input{chapters/01_introduction}

\input{chapters/02_motivation}

%%%%%%%%%%%%%%%%%%%%%%%%%%%%%%%%%%%_BIBLIOGRAPHY_%%%%%%%%%%%%%%%%%%%%%%%%%%%%%%%%%%%
% create your bibliography based on your files in library/...
% remember to edit \addbibresource in the TEMPLATE_PACKAGSES area above!
\newpage
\pagenumbering{roman} % start roman page numbers from here (optional)
\bibliographystyle{abbrv}
\bibliography{library/citations.bib}
%%%%%%%%%%%%%%%%%%%%%%%%%%%%%%%%%%%%%_APPENDIX_%%%%%%%%%%%%%%%%%%%%%%%%%%%%%%%%%%%%
\section*{Appendix} \label{sec:appendix}
\addcontentsline{toc}{section}{Appendix}    % adds entry to table of contents
\selbstaendigkeitserklaerung{\today}
%\input{chapters/xxx}                       % add in case you have additional images/tables
\end{document}
%%%%%%%%%%%%%%%%%%%%%%%%%%%%%%%%%%%%%%%%%%%%%%%%%%%%%%%%%%%%%%%%%%%%%%%%%%%%%%%%%%%%
